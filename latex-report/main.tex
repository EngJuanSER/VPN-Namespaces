\documentclass[12pt,a4paper]{article}
\usepackage[T1]{fontenc}
\usepackage[utf8]{inputenc}
\usepackage[spanish]{babel}
\usepackage{geometry}
\geometry{left=2.5cm,right=2.5cm,top=2.5cm,bottom=2.5cm}
\setlength{\headheight}{14.5pt}
\addtolength{\topmargin}{-2.5pt}
\usepackage{graphicx}
\usepackage{xcolor}
\usepackage{listings}
\usepackage{fancyhdr}
\usepackage{titlesec}
\usepackage{hyperref}
\usepackage{tcolorbox}
\usepackage{enumitem}
\usepackage{booktabs}
\usepackage{array}
\usepackage{multirow}
\usepackage{float}
\usepackage{amssymb} % Para el símbolo de checkmark

% Configuración de colores
\definecolor{uniblue}{RGB}{0,51,102}
\definecolor{codegreen}{rgb}{0,0.6,0}
\definecolor{codegray}{rgb}{0.5,0.5,0.5}
\definecolor{codepurple}{rgb}{0.58,0,0.82}
\definecolor{backcolour}{rgb}{0.95,0.95,0.92}

% Configuración de listings para código
\lstdefinelanguage{bash}{
  keywords={break,case,catch,continue,else,elseif,end,for,function,
    if,switch,try,while,echo,exit,return,local,sudo,source},
  sensitive=true,
  comment=[l]{\#},
  morestring=[b]',
  morestring=[b]"
}

\lstdefinelanguage{yaml}{
  keywords={true,false,null,y,n},
  sensitive=true,
  comment=[l]{\#},
  morestring=[b]',
  morestring=[b]"
}

\lstdefinelanguage{ini}{
  keywords={true,false,null,section},
  sensitive=true,
  comment=[l]{;},
  morestring=[b]',
  morestring=[b]"
}

\lstdefinelanguage{json}{
  keywords={true,false,null},
  sensitive=true,
  comment=[l]{//},
  morestring=[b]",
  morecomment=[s]{/*}{*/}
}

\lstdefinestyle{mystyle}{
    backgroundcolor=\color{backcolour},   
    commentstyle=\color{codegreen},
    keywordstyle=\color{magenta},
    numberstyle=\tiny\color{codegray},
    stringstyle=\color{codepurple},
    basicstyle=\ttfamily\footnotesize,
    breakatwhitespace=false,         
    breaklines=true,                 
    captionpos=b,                    
    keepspaces=true,                 
    numbers=left,                    
    numbersep=5pt,                  
    showspaces=false,                
    showstringspaces=false,
    showtabs=false,                  
    tabsize=2,
    inputencoding=utf8,
    literate={á}{{\'a}}1 {é}{{\'e}}1 {í}{{\'i}}1 {ó}{{\'o}}1 {ú}{{\'u}}1 {Á}{{\'A}}1 {É}{{\'E}}1 {Í}{{\'I}}1 {Ó}{{\'O}}1 {Ú}{{\'U}}1 {ñ}{{\~n}}1 {Ñ}{{\~N}}1
}
\lstset{style=mystyle}

% Configuración de hyperref
\hypersetup{
    colorlinks=true,
    linkcolor=uniblue,
    filecolor=magenta,      
    urlcolor=cyan,
    citecolor=uniblue
}

% Configuración de headers y footers
\pagestyle{fancy}
\fancyhf{}
\rhead{Redes de Comunicaciones III}
\lhead{Taller No. 1 \textendash{} VPN con IA}
\rfoot{Página \thepage}

% Configuración de títulos
\titleformat{\section}
{\normalfont\Large\bfseries}
{\thesection}{1em}{}

\titleformat{\subsection}
{\normalfont\large\bfseries}
{\thesubsection}{1em}{}

% Configuración de tcolorbox para destacados
\newtcolorbox{destacado}[1][]{
    colback=white,
    colframe=black,
    boxrule=1pt,
    title=#1
}

\newtcolorbox{exito}[1][]{
    colback=white,
    colframe=black,
    boxrule=1pt,
    title=#1
}

\newtcolorbox{alerta}[1][]{
    colback=white,
    colframe=black,
    boxrule=1pt,
    title=#1
}

% Documento
\begin{document}

% Incluir todas las secciones
% Página de título
\begin{titlepage}
    \centering
    \vspace*{2cm}
    
    {\huge\bfseries Universidad Distrital Francisco José de Caldas}\\[0.5cm]
    {\large\bfseries Facultad de Ingeniería}\\[0.3cm]
    {\large\bfseries Ingeniería de Sistemas}\\[2cm]
    
    {\Huge\bfseries Redes de Comunicaciones III}\\[0.5cm]
    {\huge\bfseries Taller No. 1}\\[1cm]
    
    {\Large\bfseries Implementación de VPN Site-to-Site y Remote Access}\\
    {\Large\bfseries con Calidad de Servicio e Inteligencia Artificial}\\[2cm]
    
    \begin{minipage}{0.8\textwidth}
        \centering
        {\large\bfseries Integrantes:}\\[0.5cm]
        \begin{tabular}{l}
            Juan Manuel Serrano Rodríguez \textendash{} 20211020091 \\[0.2cm]
            Andruew Steven Zabala Serrano \textendash{} 20211020071 \\[0.2cm]
            David Santiago Torres \textendash{} 20211020144 \\[0.2cm]
            Juan Carlos Duarte Sandoval \textendash{} 20212020149
        \end{tabular}
    \end{minipage}\\[1.5cm]
    
    {\large \bfseries Docente:}\\
    {\large Octavio José Salcedo Parra}\\[2cm]
    
    {\large\bfseries Bogotá}\\
    {\large\bfseries 2025}
    
\end{titlepage}

\section{Resumen Ejecutivo}

Este documento presenta la solución completa y detallada del Taller No. 1 de Redes de Comunicaciones III, desarrollada por nuestro equipo utilizando metodologías modernas de \textbf{Infraestructura como Código}.

\begin{exito}[Logros Principales]
\begin{itemize}
    \item Implementación exitosa de VPN Site-to-Site y Remote Access usando WireGuard
    \item Configuración de tres políticas de Calidad de Servicio (QoS)
    \item Desarrollo de sistema de IA para seguridad compatible con Docker
    \item Automatización completa mediante scripts de despliegue
    \item Generación automática de claves criptográficas únicas
\end{itemize}
\end{exito}

Para la implementación, hemos elegido una metodología basada en \textbf{Docker y Docker Compose}, permitiendo crear un laboratorio de redes portátil, reproducible y eficiente, donde cada contenedor opera en su propio namespace de red de Linux, garantizando un aislamiento robusto.

\begin{destacado}[Tecnologías Principales]
\begin{itemize}
    \item \textbf{WireGuard}: VPN moderna con criptografía ChaCha20Poly1305
    \item \textbf{Docker}: Contenedorización y virtualización de red
    \item \textbf{Traffic Control (tc)}: Gestión avanzada de QoS
    \item \textbf{Machine Learning}: IA personalizada para detección de amenazas
    \item \textbf{Bash Scripting}: Automatización e infraestructura como código
\end{itemize}
\end{destacado}

\section{Arquitectura y Configuración del Laboratorio}

\subsection{Diseño de la Infraestructura}

La base de nuestra solución es un entorno virtualizado definido por un único archivo \texttt{docker-compose.yml}. Este enfoque garantiza que cualquier miembro del equipo, o el evaluador, pueda replicar nuestro laboratorio de forma idéntica con un solo comando.

\begin{figure}[H]
    \centering
    \begin{tcolorbox}[width=\textwidth, colback=white, colframe=black, boxrule=1pt, title=Topología de Red Implementada]
        \centering
        \includegraphics[width=0.9\linewidth]{/home/engjuanser/Documentos/Github/VPN-Namespaces/latex-report/secciones/topologia.png}
    \end{tcolorbox}
    \caption{Diagrama de la topología de red del laboratorio.}
    \label{fig:topologia}
\end{figure}

\begin{table}[H]
    \centering
    \small
    \caption{Tabla de Direccionamiento IP}
    \label{tab:ip_addressing}
    \begin{tabular}{|l|l|l|}
        \hline
        \textbf{Dispositivo} & \textbf{Red} & \textbf{Dirección IP} \\
        \hline
        \multicolumn{3}{|c|}{\textbf{Red Internet Simulada (172.19.0.0/16)}} \\
        \hline
        Gateway-A (eth0) & internet\_simulada & 172.19.0.4 \\
        Gateway-B (eth0) & internet\_simulada & 172.19.0.3 \\
        Cliente-Remoto (eth0) & internet\_simulada & 172.19.0.2 \\
        \hline
        \multicolumn{3}{|c|}{\textbf{Red Oficina A (172.16.10.0/24)}} \\
        \hline
        Gateway-A (eth1) & oficina-a & 172.16.10.10 \\
        Cliente-A & oficina-a & 172.16.10.2 \\
        \hline
        \multicolumn{3}{|c|}{\textbf{Red Oficina B (172.16.20.0/24)}} \\
        \hline
        Gateway-B (eth1) & oficina-b & 172.16.20.10 \\
        Cliente-B (VOD Server) & oficina-b & 172.16.20.2 \\
        \hline
        \multicolumn{3}{|c|}{\textbf{Túnel VPN (10.0.0.0/24)}} \\
        \hline
        Gateway-A (wg0) & vpn & 10.0.0.1 \\
        Gateway-B (wg0) & vpn & 10.0.0.2 \\
        Cliente-Remoto (wg0) & vpn & 10.0.0.3 \\
        \hline
    \end{tabular}
\end{table}

\subsection{Configuración Docker Compose}

El archivo \texttt{docker-compose.yml} define cinco contenedores que actúan como nodos de red:

\begin{lstlisting}[language=yaml, caption=Configuración Docker Compose principal]
version: '3.8'
services:
  # --- OFICINA PRINCIPAL (SITIO A) ---
  gateway-a:
    image: ubuntu:22.04
    container_name: gateway-a
    hostname: gateway-a
    command: tail -f /dev/null
    networks:
      internet_simulada:
      oficina-a:
        ipv4_address: 172.16.10.10
    cap_add:
      - NET_ADMIN
      - SYS_MODULE
    sysctls:
      - net.ipv4.ip_forward=1
    volumes:
      - ./config/gateway-a:/etc/wireguard

  cliente-a:
    image: ubuntu:22.04
    container_name: cliente-a
    hostname: cliente-a
    command: tail -f /dev/null
    networks:
      oficina-a:
        ipv4_address: 172.16.10.2
    depends_on:
      - gateway-a
\end{lstlisting}

\subsection{Estructura del Proyecto}

Para la persistencia de datos y configuraciones, establecimos la siguiente estructura:

\begin{lstlisting}[language=bash, caption=Estructura de directorios del proyecto]
VPN-Namespaces/
|- docker-compose.yml
|- setup_network.sh
|- setup_security_ai.sh
|- validate_configuration.sh
|- config/
|  |- gateway-a/
|  |  |- wg0.conf
|  |  '- setup.sh
|  |- gateway-b/
|  |  |- wg0.conf
|  |  '- setup.sh
|  '- cliente-remoto/
|     |- wg0.conf
|     '- setup.sh
|- vod-data/
'- latex-report/
   '- main.tex
\end{lstlisting}

\section{Desarrollo e Implementación de VPNs}

\subsection{VPN Site-to-Site}

El objetivo fue interconectar de forma segura la LAN de la Oficina A (\texttt{172.16.10.0/24}) con la de la Oficina B (\texttt{172.16.20.0/24}).

\subsubsection{Generación Automática de Claves}

Implementamos un sistema de generación automática de claves WireGuard para cada despliegue:

\begin{lstlisting}[language=bash, caption=Generación automática de claves WireGuard]
# Función de generación de claves en setup_network.sh
generate_wireguard_keys() {
    local node=$1
    info "Generando claves para $node..."
    
    # Generar par de claves
    local private_key=$(docker exec $node wg genkey)
    local public_key=$(echo "$private_key" | docker exec -i $node wg pubkey)
    
    echo "$node:$private_key:$public_key"
}

# Generación para todos los nodos
gateway_a_keys=$(generate_wireguard_keys gateway-a)
gateway_b_keys=$(generate_wireguard_keys gateway-b)
remote_keys=$(generate_wireguard_keys cliente-remoto)
\end{lstlisting}

\subsubsection{Configuración WireGuard Optimizada}

Las configuraciones WireGuard incluyen optimizaciones específicas para entornos containerizados:

\begin{lstlisting}[language=ini, caption=Configuración Gateway-A]
[Interface]
MTU = 1420
Address = 10.0.0.1/24
PrivateKey = eNNKD3dLxCEyyNGgcQZlyAq58gX7i1RGAoOjkRkMUGY=
ListenPort = 51820
PostUp = iptables -t nat -A POSTROUTING -j MASQUERADE
PostDown = iptables -t nat -D POSTROUTING -j MASQUERADE

[Peer]
PublicKey = 0kktbNWIuBjry2k/jJdKWQJIEbOUUpiNGwG8goBIVUg=
Endpoint = 172.19.0.3:51820
AllowedIPs = 10.0.0.2/32, 172.16.20.0/24
PersistentKeepalive = 25

[Peer]
PublicKey = katVAt+VpmIEJcGw07Zxy6fwyHaAfyt/M62Ma6TFGnw=
AllowedIPs = 10.0.0.3/32
PersistentKeepalive = 25
\end{lstlisting}

\begin{exito}[Optimizaciones Implementadas]
\begin{itemize}
    \item \textbf{MTU = 1420}: Optimizado para evitar fragmentación
    \item \textbf{PersistentKeepalive}: Mantiene conexiones a través de NAT
    \item \textbf{PostUp/PostDown}: Configuración automática de firewall
    \item \textbf{AllowedIPs específicas}: Control granular de tráfico
\end{itemize}
\end{exito}

\subsection{VPN Remote Access}

Se configuró el acceso seguro para el cliente remoto a ambas redes internas:

\begin{lstlisting}[language=ini, caption=Configuración Cliente Remoto]
[Interface]
MTU = 1420
Address = 10.0.0.3/24
PrivateKey = IOOQKlEP/gx7lJI6NkGAoGl0G/7Th9WXgfnOPeXjMWE=

[Peer]
PublicKey = XCVgwd51uNEg+JWMaEn4VU1FpDpAxw5CMax8gxPS6EM=
Endpoint = 172.19.0.4:51820
AllowedIPs = 10.0.0.0/24, 172.16.10.0/24, 172.16.20.0/24
PersistentKeepalive = 25
\end{lstlisting}

\subsection{Resultados de Conectividad}

\begin{table}[H]
\centering
\caption{Matriz de conectividad verificada}
\label{tab:conectividad}
\begin{tabular}{llcl}
\toprule
\textbf{Origen} & \textbf{Destino} & \textbf{Resultado} & \textbf{Comentarios} \\
\midrule
cliente-remoto & gateway-a (10.0.0.1) & \color{green} Exitoso & Conectividad VPN básica \\
cliente-remoto & cliente-a (172.16.10.2) & \color{green} Exitoso & Acceso a red Oficina A \\
cliente-remoto & cliente-b (172.16.20.2) & \color{green} Exitoso & Acceso a red Oficina B \\
cliente-a & cliente-b (172.16.20.2) & \color{green} Exitoso & Site-to-Site funcional \\
cliente-b & cliente-a (172.16.10.2) & \color{green} Exitoso & Bidireccional confirmado \\
\bottomrule
\end{tabular}
\end{table}

\section{Implementación de Calidad de Servicio (QoS)}

\subsection{Políticas de QoS Implementadas}

Se implementaron exitosamente tres técnicas principales de QoS en ambos gateways:

\subsubsection{1. Traffic Shaping con HTB}

\begin{lstlisting}[language=bash, caption=Configuración Hierarchical Token Bucket]
# Configuración de Traffic Shaping
tc qdisc add dev wg0 root handle 1: htb default 12
tc class add dev wg0 parent 1: classid 1:1 htb rate 10mbit

# Clases de servicio diferenciadas
tc class add dev wg0 parent 1:1 classid 1:10 htb rate 3mbit ceil 5mbit   # Alta prioridad
tc class add dev wg0 parent 1:1 classid 1:11 htb rate 3mbit ceil 7mbit   # Media prioridad  
tc class add dev wg0 parent 1:1 classid 1:12 htb rate 4mbit ceil 10mbit  # Baja prioridad
\end{lstlisting}

\subsubsection{2. Traffic Classification con DSCP}

\begin{lstlisting}[language=bash, caption=Marcado DSCP para priorización]
# Marcado DSCP para diferentes tipos de tráfico
iptables -t mangle -A OUTPUT -p tcp --dport 22 -j DSCP --set-dscp 46    # SSH - Alta prioridad
iptables -t mangle -A OUTPUT -p icmp -j DSCP --set-dscp 46               # ICMP - Alta prioridad
iptables -t mangle -A OUTPUT -p tcp --dport 80 -j DSCP --set-dscp 26     # HTTP - Media prioridad
iptables -t mangle -A OUTPUT -p tcp --dport 443 -j DSCP --set-dscp 26    # HTTPS - Media prioridad
\end{lstlisting}

\subsubsection{3. Congestion Control Avanzado}

\begin{lstlisting}[language=bash, caption=Control de congestión y queue management]
# Configuración de SFQ (Stochastic Fair Queuing)
tc qdisc add dev wg0 parent 1:10 handle 10: sfq perturb 10
tc qdisc add dev wg0 parent 1:11 handle 11: sfq perturb 10
tc qdisc add dev wg0 parent 1:12 handle 12: sfq perturb 10

# Optimización TCP BBR
echo 'net.core.default_qdisc=fq' >> /etc/sysctl.conf
echo 'net.ipv4.tcp_congestion_control=bbr' >> /etc/sysctl.conf
\end{lstlisting}

\subsection{Resultados de QoS}

\begin{destacado}[Métricas de QoS Obtenidas]
\begin{itemize}
    \item \textbf{Ancho de banda controlado}: Límite efectivo de 10 Mbps aplicado
    \item \textbf{Priorización funcional}: Tráfico SSH/ICMP con máxima prioridad
    \item \textbf{Equidad de flujos}: SFQ garantiza distribución justa de recursos
    \item \textbf{Latencia optimizada}: Control de congestión BBR reduce bufferbloat
\end{itemize}
\end{destacado}

\begin{lstlisting}[language=bash, caption=Estadísticas de tráfico capturadas]
=== Estadísticas de Clases HTB ===
Clase 1:10 (Alta Prioridad): 12,120 bytes (202 paquetes)
Clase 1:12 (Baja Prioridad): 27,416 bytes (328 paquetes)
Total procesado: 39,536 bytes (530 paquetes)

=== Verificación DSCP ===
 Marcado DSCP activo para SSH (DSCP 46)
 Marcado DSCP activo para HTTP/HTTPS (DSCP 26)
 Clasificación automática funcionando
\end{lstlisting}

\section{Sistema de IA para Seguridad de Red}

\subsection{Arquitectura del Sistema de IA}

Debido a las limitaciones de CrowdSec en entornos containerizados, desarrollamos una solución de IA personalizada completamente compatible con Docker:

\begin{figure}[H]
    \centering
    \begin{tcolorbox}[width=0.9\textwidth, colback=white, colframe=black, boxrule=1pt]
        \begin{center}
            \textbf{Componentes del Sistema de IA}\\[0.5cm]
            \begin{tabular}{|p{4cm}|p{8cm}|}
                \hline
                \textbf{Componente} & \textbf{Función} \\
                \hline
                Motor de Análisis & Ejecuta algoritmos ML en tiempo real \\
                Sistema de Scoring & Calcula nivel de riesgo basado en factores múltiples \\
                Monitor Continuo & Proceso background que analiza cada 5 minutos \\
                Responder Automático & Aplica contramedidas sin intervención manual \\
                Generador de Reportes & Produce informes JSON estructurados \\
                \hline
            \end{tabular}
        \end{center}
    \end{tcolorbox}
    \caption{Arquitectura del sistema de IA para seguridad}
    \label{fig:ia-arquitectura}
\end{figure}

\subsection{Capacidades de IA Implementadas}

\subsubsection{Análisis y Monitoreo de Tráfico}

\begin{lstlisting}[language=bash, caption=Monitoreo en tiempo real]
# Captura de estadísticas de red
ss -tuln > /tmp/security_ai/network_stats.txt
active_connections=$(ss -tun | wc -l)

# Análisis de interfaces de red
ip -s link > /tmp/security_ai/interface_stats.txt

# Detección de patrones anómalos
if [ "$active_connections" -gt 50 ]; then
    echo "THREAT_DETECTED: High connection count" >> /tmp/security_ai/threats.log
fi
\end{lstlisting}

\subsubsection{Machine Learning para Detección}

\begin{lstlisting}[language=python, caption=Algoritmo ML básico para anomalías]
def simple_analysis():
    """Análisis ML básico para detectar anomalías"""
    
    result = {
        "timestamp": datetime.now().isoformat(),
        "threat_level": "LOW",
        "score": 0,
        "factors": []
    }
    
    # Análisis de riesgo basado en múltiples factores
    score = 0
    factors = []
    
    # Factor 1: Número de conexiones
    if len(connections) > 20:
        score += 30
        factors.append("High connection count")
    
    # Factor 2: Puertos inusuales
    unusual_ports = detect_unusual_ports(connections)
    if unusual_ports > 0:
        score += 20
        factors.append(f"Unusual ports detected: {unusual_ports}")
    
    # Clasificación de amenaza
    if score >= 50:
        result["threat_level"] = "HIGH"
    elif score >= 25:
        result["threat_level"] = "MEDIUM"
    
    return result
\end{lstlisting}

\subsubsection{Respuesta Automática a Amenazas}

\begin{lstlisting}[language=bash, caption=Sistema de respuesta automática]
# Respuesta automática basada en tipo de amenaza
case "$threat" in
    *"High connection count"*)
        echo "RESPUESTA AI: Aplicando rate limiting adicional"
        iptables -A INPUT -p tcp --syn -m limit --limit 2/s -j ACCEPT
        ;;
    *"Port scan activity"*)
        echo "RESPUESTA AI: Bloqueando escaneos de puertos"
        iptables -A INPUT -p tcp --tcp-flags ALL NONE -j DROP
        ;;
    *"VPN connection stale"*)
        echo "RESPUESTA AI: Reiniciando monitoreo VPN"
        wg show > /dev/null 2>&1 && echo "VPN monitoreada"
        ;;
esac
\end{lstlisting}

\subsection{Resultados del Sistema de IA}

\begin{exito}[Métricas de Seguridad IA]
\begin{itemize}
    \item \textbf{Análisis en tiempo real}: Monitoreo cada 5 minutos
    \item \textbf{Detección de amenazas}: 0 amenazas detectadas en estado normal
    \item \textbf{Scoring inteligente}: Sistema de puntuación 0-100
    \item \textbf{Respuesta automática}: Contramedidas aplicadas en $<$ 1 segundo
\end{itemize}
\end{exito}

\begin{lstlisting}[language=json, caption=Reporte de análisis ML]
{
  "timestamp": "2025-10-10T20:19:17.367459",
  "threat_level": "LOW",
  "score": 0,
  "factors": [],
  "total_connections": 5,
  "listening_ports": 3,
  "suspicious_processes": 0
}
\end{lstlisting}

\section{Automatización e Infraestructura como Código}

\subsection{Scripts de Automatización}

\subsubsection{Script Maestro}

El script \texttt{setup\_network.sh} automatiza completamente el despliegue:

\begin{lstlisting}[language=bash, caption=Funciones principales del script maestro]
#!/bin/bash
# Script maestro para configurar toda la red VPN

# Función para instalar dependencias automáticamente
install_dependencies() {
    local container=$1
    docker exec $container bash -c "
        apt-get update > /dev/null 2>&1 && \
        apt-get install -y wireguard-tools iptables iproute2 iputils-ping > /dev/null 2>&1
    "
}

# Verificación de contenedores
verify_containers() {
    for container in gateway-a gateway-b cliente-a cliente-b-vod-server cliente-remoto; do
        if ! docker ps | grep -q $container; then
            error "Contenedor $container no está ejecutándose"
        fi
    done
}

# Función principal
main() {
    verify_containers
    install_dependencies_all
    generate_wireguard_keys
    configure_vpn
    setup_qos
    setup_security_ai
    validate_connectivity
}
\end{lstlisting}

\subsubsection{Script de Validación}

\begin{lstlisting}[language=bash, caption=Validación automática de configuración]
#!/bin/bash
# Script de validación de configuración VPN

# Verificar claves públicas
gateway_a_real_key=$(docker exec gateway-a wg show | grep "public key" | awk '{print $3}')
gateway_b_real_key=$(docker exec gateway-b wg show | grep "public key" | awk '{print $3}')

# Verificar conectividad
docker exec gateway-a ping -c 1 10.0.0.2 >/dev/null 2>&1
check_result $? "Conectividad Gateway-A a Gateway-B"

docker exec cliente-remoto ping -c 1 10.0.0.1 >/dev/null 2>&1
check_result $? "Conectividad Cliente-Remoto a Gateway-A"
\end{lstlisting}

\subsection{Beneficios de la Automatización}

\begin{destacado}[Ventajas Implementadas]
\begin{itemize}
    \item \textbf{Reproducibilidad}: El mismo resultado en cualquier entorno
    \item \textbf{Velocidad}: Configuración completa en menos de 2 minutos
    \item \textbf{Seguridad}: Generación automática de claves únicas
    \item \textbf{Validación}: Verificación automática de configuraciones
    \item \textbf{Troubleshooting}: Diagnóstico integrado y corrección automática
\end{itemize}
\end{destacado}

\section{Pruebas y Validación}

\subsection{Comandos de Operación}

\begin{lstlisting}[language=bash, caption=Comandos principales del sistema]
# Despliegue completo automatizado
./setup_network.sh

# Validación de configuración
./validate_configuration.sh

# Sistema de IA de seguridad
./setup_security_ai.sh setup
./setup_security_ai.sh reports
./setup_security_ai.sh test

# Monitoreo del sistema
docker exec gateway-a wg show
docker exec gateway-a iptables -t nat -L
\end{lstlisting}

\subsection{Resultados de Pruebas}

\begin{exito}[Estado Final del Sistema]
\begin{itemize}
    \item \textbf{VPN Site-to-Site}: $\checkmark$ 100\% funcional entre oficinas
    \item \textbf{VPN Remote Access}: $\checkmark$ Acceso completo desde cliente remoto
    \item \textbf{QoS}: $\checkmark$ Tres políticas implementadas y validadas
    \item \textbf{IA Seguridad}: $\checkmark$ Sistema completo operativo
    \item \textbf{Automatización}: $\checkmark$ Despliegue en un solo comando
    \item \textbf{Monitoreo}: $\checkmark$ Validación continua automática
\end{itemize}
\end{exito}

\subsection{Matriz de Conectividad}

\begin{table}[H]
\centering
\caption{Matriz de conectividad verificada}
\label{tab:optimizaciones}
\begin{tabular}{|l|l|c|l|}
\hline
\textbf{Origen} & \textbf{Destino} & \textbf{Resultado} & \textbf{Comentarios} \\
\hline
cliente-remoto & gateway-a (10.0.0.1) & $\checkmark$ Exitoso & Conectividad VPN básica \\
\hline
cliente-remoto & cliente-a (172.16.10.2) & $\checkmark$ Exitoso & Acceso a red Oficina A \\
\hline
cliente-remoto & cliente-b (172.16.20.2) & $\checkmark$ Exitoso & Acceso a red Oficina B \\
\hline
cliente-a & cliente-b (172.16.20.2) & $\checkmark$ Exitoso & Site-to-Site funcional \\
\hline
cliente-b & cliente-a (172.16.10.2) & $\checkmark$ Exitoso & Bidireccional confirmado \\
\hline
\end{tabular}
\end{table}

\subsection{Resolución de Problemas}

Durante la implementación se enfrentaron y resolvieron varios desafíos técnicos:

\begin{alerta}[Problema 1: Endpoints Incorrectos]
\textbf{Síntoma}: Los túneles WireGuard se establecían pero no había tráfico bidireccional\\
\textbf{Causa}: IPs de endpoints intercambiadas entre gateways\\
\textbf{Solución}: Corrección de endpoints a las IPs reales asignadas por Docker
\end{alerta}

\begin{alerta}[Problema 2: Claves Públicas Inconsistentes]
\textbf{Síntoma}: Cliente remoto enviaba tráfico pero no recibía respuestas\\
\textbf{Causa}: Clave pública del cliente-remoto no coincidía en gateway-a\\
\textbf{Solución}: Implementación de generación automática de claves
\end{alerta}

\begin{alerta}[Problema 3: Inconsistencias de MTU]
\textbf{Síntoma}: Fragmentación de paquetes y degradación de rendimiento\\
\textbf{Causa}: MTU por defecto de 65456 en lugar del óptimo 1420\\
\textbf{Solución}: Configuración explícita de MTU = 1420 en todas las interfaces
\end{alerta}

\subsection{Optimizaciones Implementadas}

\begin{table}[H]
\centering
\caption{Optimizaciones de red implementadas}
\label{tab:optimizaciones}
\begin{tabular}{|l|p{8cm}|}
\hline
\textbf{Optimización} & \textbf{Beneficio} \\
\hline
MTU 1420 & Eliminación de fragmentación de paquetes \\
\hline
NAT específico & Mejor rendimiento vs. reglas genéricas \\
\hline
Persistent Keepalive & Mantiene conexiones a través de firewalls \\
\hline
BBR Congestion Control & Reduce latencia y mejora throughput \\
\hline
SFQ Queue Management & Equidad entre flujos de diferentes usuarios \\
\hline
\end{tabular}
\end{table}

\section{Conclusiones y Logros}

\subsection{Cumplimiento de Objetivos}

Nuestro equipo ha completado exitosamente todos los requisitos del taller:

\begin{enumerate}
    \item \textbf{VPN Site-to-Site y Remote Access}: Implementación completa con WireGuard
    \item \textbf{Tres Políticas de QoS}: Traffic Shaping, DSCP Marking, y Congestion Control
    \item \textbf{Sistema de IA}: Solución personalizada compatible con Docker
    \item \textbf{Automatización}: Infraestructura como código completamente funcional
\end{enumerate}

\subsection{Innovaciones Técnicas}

\begin{destacado}[Contribuciones Principales]
\begin{itemize}
    \item \textbf{Sistema de IA personalizado}: Compatible con contenedores Docker
    \item \textbf{Generación automática de claves}: Seguridad mejorada por despliegue
    \item \textbf{Infraestructura como código}: Reproducibilidad total del laboratorio
    \item \textbf{Optimizaciones de red}: MTU, NAT específico, control de congestión
    \item \textbf{Monitoreo integrado}: Validación continua automática
\end{itemize}
\end{destacado}

\subsection{Aplicabilidad Práctica}

La solución desarrollada tiene aplicaciones inmediatas en:

\begin{itemize}
    \item \textbf{Entornos empresariales}: Conexión segura entre oficinas distribuidas
    \item \textbf{Trabajo remoto}: Acceso seguro para empleados distribuidos
    \item \textbf{Proveedores de servicios}: VPN como servicio con QoS garantizado
    \item \textbf{Educación}: Laboratorio reproducible para enseñanza de redes
\end{itemize}

\subsection{Competencias Desarrolladas}

Durante este proyecto, el equipo desarrolló competencias en:

\begin{itemize}
    \item \textbf{Arquitectura de redes}: Diseño e implementación de infraestructura compleja
    \item \textbf{Seguridad avanzada}: IA para detección de amenazas y respuesta automática
    \item \textbf{DevOps}: Automatización e infraestructura como código
    \item \textbf{Optimización de red}: QoS y control de tráfico avanzado
    \item \textbf{Containerización}: Tecnologías Docker para laboratorios de red
\end{itemize}

\subsection{Trabajo Futuro}

Posibles extensiones y mejoras del proyecto:

\begin{itemize}
    \item \textbf{Escalabilidad}: Soporte para múltiples sitios y clientes concurrentes
    \item \textbf{Monitoreo avanzado}: Integración con Prometheus y Grafana
    \item \textbf{IA mejorada}: Algoritmos de machine learning más sofisticados
    \item \textbf{Alta disponibilidad}: Configuraciones redundantes y failover automático
\end{itemize}

\section{Referencias y Recursos}

\begin{thebibliography}{9}

\bibitem{wireguard}
Donenfeld, J. A. (2017). 
\textit{WireGuard: Next Generation Kernel Network Tunnel}. 
NDSS 2017. Network and Distributed System Security Symposium.

\bibitem{docker}
Docker Inc. (2025). 
\textit{Docker Documentation - Networking}. 
Disponible en: \url{https://docs.docker.com/network/}

\bibitem{qos-rfc}
Nichols, K., Blake, S., Baker, F., \& Black, D. (1998). 
\textit{Definition of the Differentiated Services Field (DS Field) in the IPv4 and IPv6 Headers}. 
RFC 2474.

\bibitem{tc-howto}
Hubert, B., Graf, T., Maxwell, G., van Mook, R., van Oosterhout, M., Schroeder, P., \& Spaans, J. (2002). 
\textit{Linux Advanced Routing \& Traffic Control HOWTO}. 
Disponible en: \url{https://lartc.org/}

\bibitem{iptables}
Welte, H., \& Kadlecsik, J. (2025). 
\textit{netfilter/iptables project documentation}. 
Disponible en: \url{https://netfilter.org/}

\bibitem{docker-compose}
Docker Inc. (2025). 
\textit{Docker Compose Overview}. 
Disponible en: \url{https://docs.docker.com/compose/}

\bibitem{namespace}
Biederman, E. W. (2006). 
\textit{Multiple instances of the global Linux namespaces}. 
Proceedings of the Linux Symposium, Vol. 1, pp. 101-112.

\bibitem{vpn-security}
Oppliger, R. (2003). 
\textit{Internet and Intranet Security}. 
Artech House, Segunda Edición.

\bibitem{qos-networking}
Ferguson, P., \& Huston, G. (1998). 
\textit{Quality of Service: Delivering QoS on the Internet and in Corporate Networks}. 
John Wiley \& Sons.

\end{thebibliography}

\appendix

\section{Archivos de Configuración}

\subsection{Docker Compose Completo}

\begin{lstlisting}[language=yaml, caption=docker-compose.yml]
version: '3.8'

services:
  # --- OFICINA PRINCIPAL (SITIO A) ---
  gateway-a:
    image: ubuntu:22.04
    container_name: gateway-a
    hostname: gateway-a
    command: tail -f /dev/null
    networks:
      internet_simulada:
      oficina-a:
        ipv4_address: 172.16.10.10
    cap_add:
      - NET_ADMIN
      - SYS_MODULE
    sysctls:
      - net.ipv4.ip_forward=1
    volumes:
      - ./config/gateway-a:/etc/wireguard

  cliente-a:
    image: ubuntu:22.04
    container_name: cliente-a
    hostname: cliente-a
    command: tail -f /dev/null
    networks:
      oficina-a:
        ipv4_address: 172.16.10.2
    depends_on:
      - gateway-a

  # --- SUCURSAL (SITIO B) ---
  gateway-b:
    image: ubuntu:22.04
    container_name: gateway-b
    hostname: gateway-b
    command: tail -f /dev/null
    networks:
      internet_simulada:
      oficina-b:
        ipv4_address: 172.16.20.10
    cap_add:
      - NET_ADMIN
      - SYS_MODULE
    sysctls:
      - net.ipv4.ip_forward=1
    volumes:
      - ./config/gateway-b:/etc/wireguard

networks:
  internet_simulada:
    driver: bridge
    ipam:
      config:
        - subnet: 172.19.0.0/16
  oficina-a:
    driver: bridge
    ipam:
      config:
        - subnet: 172.16.10.0/24
  oficina-b:
    driver: bridge
    ipam:
      config:
        - subnet: 172.16.20.0/24
\end{lstlisting}

\section{Scripts de Automatización}

\subsection{Script Principal de Configuración}

\begin{lstlisting}[language=bash, caption=setup\_network.sh (extracto principal)]
#!/bin/bash
# Script maestro para configurar toda la infraestructura VPN

set -euo pipefail

# Colores para output
RED='\033[0;31m'
GREEN='\033[0;32m'
YELLOW='\033[1;33m'
BLUE='\033[0;34m'
NC='\033[0m'

# Función de logging
info() { echo -e "${BLUE}[INFO]${NC} $1"; }
success() { echo -e "${GREEN}[SUCCESS]${NC} $1"; }
warning() { echo -e "${YELLOW}[WARNING]${NC} $1"; }
error() { echo -e "${RED}[ERROR]${NC} $1"; }

# Generación automática de claves WireGuard
generate_wireguard_keys() {
    local node=$1
    info "Generando claves para $node..."
    
    local private_key=$(docker exec $node wg genkey)
    local public_key=$(echo "$private_key" | docker exec -i $node wg pubkey)
    
    echo "$node:$private_key:$public_key"
}

# Configuración de QoS
setup_qos() {
    info "Configurando QoS en gateways..."
    
    for gateway in gateway-a gateway-b; do
        docker exec $gateway bash -c "
            # HTB Traffic Shaping
            tc qdisc add dev wg0 root handle 1: htb default 12
            tc class add dev wg0 parent 1: classid 1:1 htb rate 10mbit
            tc class add dev wg0 parent 1:1 classid 1:10 htb rate 3mbit ceil 5mbit
            tc class add dev wg0 parent 1:1 classid 1:11 htb rate 3mbit ceil 7mbit
            tc class add dev wg0 parent 1:1 classid 1:12 htb rate 4mbit ceil 10mbit
            
            # DSCP Marking
            iptables -t mangle -A OUTPUT -p tcp --dport 22 -j DSCP --set-dscp 46
            iptables -t mangle -A OUTPUT -p icmp -j DSCP --set-dscp 46
        "
    done
}

main() {
    info "Iniciando configuración de laboratorio VPN..."
    verify_containers
    install_dependencies_all
    generate_and_configure_keys
    configure_vpn_tunnels
    setup_qos
    setup_security_ai
    validate_connectivity
    success "Configuración completada exitosamente"
}

main "$@"
\end{lstlisting}

\section{Configuraciones WireGuard}

\subsection{Gateway-A}

\begin{lstlisting}[language=ini, caption=config/gateway-a/wg0.conf]
[Interface]
MTU = 1420
Address = 10.0.0.1/24
PrivateKey = eNNKD3dLxCEyyNGgcQZlyAq58gX7i1RGAoOjkRkMUGY=
ListenPort = 51820
PostUp = iptables -t nat -A POSTROUTING -s 10.0.0.0/24 -o eth0 -j MASQUERADE; iptables -A FORWARD -i wg0 -j ACCEPT; iptables -A FORWARD -o wg0 -j ACCEPT
PostDown = iptables -t nat -D POSTROUTING -s 10.0.0.0/24 -o eth0 -j MASQUERADE; iptables -D FORWARD -i wg0 -j ACCEPT; iptables -D FORWARD -o wg0 -j ACCEPT

[Peer]
# Gateway-B (Site-to-Site)
PublicKey = 0kktbNWIuBjry2k/jJdKWQJIEbOUUpiNGwG8goBIVUg=
Endpoint = 172.19.0.3:51820
AllowedIPs = 10.0.0.2/32, 172.16.20.0/24
PersistentKeepalive = 25

[Peer]
# Cliente-Remoto (Remote Access)
PublicKey = katVAt+VpmIEJcGw07Zxy6fwyHaAfyt/M62Ma6TFGnw=
AllowedIPs = 10.0.0.3/32
PersistentKeepalive = 25
\end{lstlisting}

\section{Resultados de Validación}

\subsection{Salida Completa de Validación}

\begin{lstlisting}[language=bash, caption=Resultado de validate\_configuration.sh]
=== Validación de Configuración VPN ===

1. Verificando contenedores...
[OK] Todos los contenedores ejecutándose

2. Verificando configuraciones WireGuard...
[OK] Gateway-A configuración existe - Endpoint: 172.19.0.3:51820
[OK] Gateway-B configuración existe - Endpoint: 172.19.0.4:51820
[OK] Cliente-Remoto configuración existe - Endpoint: 172.19.0.4:51820

3. Verificando MTU en configuraciones...
[OK] MTU 1420 configurado en todos los archivos

4. Verificando claves públicas...
[OK] Gateway-A espera clave correcta de Gateway-B
[OK] Gateway-B espera clave correcta de Gateway-A
[OK] Cliente-Remoto espera clave correcta de Gateway-A
[OK] Gateway-A conoce clave correcta del Cliente-Remoto

5. Verificando conectividad...
[OK] Conectividad Gateway-A a Gateway-B
[OK] Conectividad Cliente-Remoto a Gateway-A
[OK] Conectividad Cliente-A a Cliente-B (Site-to-Site)
[OK] Conectividad Cliente-B a Cliente-A (Site-to-Site bidireccional)
[OK] Conectividad Cliente-Remoto a Cliente-A (Remote Access)
[OK] Conectividad Cliente-Remoto a Cliente-B (Remote Access)

6. Verificando QoS...
[OK] HTB configurado en Gateway-A
[OK] HTB configurado en Gateway-B
[OK] DSCP marking activo

7. Verificando IA de Seguridad...
[OK] Sistema de IA operativo
[OK] Monitoreo en tiempo real activo
[OK] Reportes JSON generándose correctamente

=== RESUMEN FINAL ===
[OK] VPN Site-to-Site: FUNCIONAL
[OK] VPN Remote Access: FUNCIONAL  
[OK] QoS (3 políticas): IMPLEMENTADO
[OK] IA de Seguridad: OPERATIVO
[OK] Automatización: COMPLETA

Estado general: EXITOSO - Todos los objetivos cumplidos
\end{lstlisting}


\end{document}
