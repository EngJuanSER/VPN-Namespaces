\section{Desarrollo e Implementación de VPNs}

\subsection{VPN Site-to-Site}

El objetivo fue interconectar de forma segura la LAN de la Oficina A (\texttt{172.16.10.0/24}) con la de la Oficina B (\texttt{172.16.20.0/24}).

\subsubsection{Generación Automática de Claves}

Implementamos un sistema de generación automática de claves WireGuard para cada despliegue:

\begin{lstlisting}[language=bash, caption=Generación automática de claves WireGuard]
# Función de generación de claves en setup_network.sh
generate_wireguard_keys() {
    local node=$1
    info "Generando claves para $node..."
    
    # Generar par de claves
    local private_key=$(docker exec $node wg genkey)
    local public_key=$(echo "$private_key" | docker exec -i $node wg pubkey)
    
    echo "$node:$private_key:$public_key"
}

# Generación para todos los nodos
gateway_a_keys=$(generate_wireguard_keys gateway-a)
gateway_b_keys=$(generate_wireguard_keys gateway-b)
remote_keys=$(generate_wireguard_keys cliente-remoto)
\end{lstlisting}

\subsubsection{Configuración WireGuard Optimizada}

Las configuraciones WireGuard incluyen optimizaciones específicas para entornos containerizados:

\begin{lstlisting}[language=ini, caption=Configuración Gateway-A]
[Interface]
MTU = 1420
Address = 10.0.0.1/24
PrivateKey = eNNKD3dLxCEyyNGgcQZlyAq58gX7i1RGAoOjkRkMUGY=
ListenPort = 51820
PostUp = iptables -t nat -A POSTROUTING -j MASQUERADE
PostDown = iptables -t nat -D POSTROUTING -j MASQUERADE

[Peer]
PublicKey = 0kktbNWIuBjry2k/jJdKWQJIEbOUUpiNGwG8goBIVUg=
Endpoint = 172.19.0.3:51820
AllowedIPs = 10.0.0.2/32, 172.16.20.0/24
PersistentKeepalive = 25

[Peer]
PublicKey = katVAt+VpmIEJcGw07Zxy6fwyHaAfyt/M62Ma6TFGnw=
AllowedIPs = 10.0.0.3/32
PersistentKeepalive = 25
\end{lstlisting}

\begin{exito}[Optimizaciones Implementadas]
\begin{itemize}
    \item \textbf{MTU = 1420}: Optimizado para evitar fragmentación
    \item \textbf{PersistentKeepalive}: Mantiene conexiones a través de NAT
    \item \textbf{PostUp/PostDown}: Configuración automática de firewall
    \item \textbf{AllowedIPs específicas}: Control granular de tráfico
\end{itemize}
\end{exito}

\subsection{VPN Remote Access}

Se configuró el acceso seguro para el cliente remoto a ambas redes internas:

\begin{lstlisting}[language=ini, caption=Configuración Cliente Remoto]
[Interface]
MTU = 1420
Address = 10.0.0.3/24
PrivateKey = IOOQKlEP/gx7lJI6NkGAoGl0G/7Th9WXgfnOPeXjMWE=

[Peer]
PublicKey = XCVgwd51uNEg+JWMaEn4VU1FpDpAxw5CMax8gxPS6EM=
Endpoint = 172.19.0.4:51820
AllowedIPs = 10.0.0.0/24, 172.16.10.0/24, 172.16.20.0/24
PersistentKeepalive = 25
\end{lstlisting}

\subsection{Resultados de Conectividad}

\begin{table}[H]
\centering
\caption{Matriz de conectividad verificada}
\label{tab:conectividad}
\begin{tabular}{llcl}
\toprule
\textbf{Origen} & \textbf{Destino} & \textbf{Resultado} & \textbf{Comentarios} \\
\midrule
cliente-remoto & gateway-a (10.0.0.1) & \color{green} Exitoso & Conectividad VPN básica \\
cliente-remoto & cliente-a (172.16.10.2) & \color{green} Exitoso & Acceso a red Oficina A \\
cliente-remoto & cliente-b (172.16.20.2) & \color{green} Exitoso & Acceso a red Oficina B \\
cliente-a & cliente-b (172.16.20.2) & \color{green} Exitoso & Site-to-Site funcional \\
cliente-b & cliente-a (172.16.10.2) & \color{green} Exitoso & Bidireccional confirmado \\
\bottomrule
\end{tabular}
\end{table}
