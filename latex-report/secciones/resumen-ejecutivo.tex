\section{Resumen Ejecutivo}

Este documento presenta la solución completa y detallada del Taller No. 1 de Redes de Comunicaciones III, desarrollada por nuestro equipo utilizando metodologías modernas de \textbf{Infraestructura como Código}.
Todo el codigo quedo contenido en el repositorio de GitHub: \url{https://github.com/EngJuanSER/VPN-Namespaces}
\begin{exito}[Logros Principales]
\begin{itemize}
    \item Implementación exitosa de VPN Site-to-Site y Remote Access usando WireGuard
    \item Configuración de tres políticas de Calidad de Servicio (QoS)
    \item Desarrollo de sistema de IA para seguridad compatible con Docker
    \item Automatización completa mediante scripts de despliegue
    \item Generación automática de claves criptográficas únicas
\end{itemize}
\end{exito}

Para la implementación, hemos elegido una metodología basada en \textbf{Docker y Docker Compose}, permitiendo crear un laboratorio de redes portátil, reproducible y eficiente, donde cada contenedor opera en su propio namespace de red de Linux, garantizando un aislamiento robusto.

\begin{destacado}[Tecnologías Principales]
\begin{itemize}
    \item \textbf{WireGuard}: VPN moderna con criptografía ChaCha20Poly1305
    \item \textbf{Docker}: Contenedorización y virtualización de red
    \item \textbf{Traffic Control (tc)}: Gestión avanzada de QoS
    \item \textbf{Machine Learning}: IA personalizada para detección de amenazas
    \item \textbf{Bash Scripting}: Automatización e infraestructura como código
\end{itemize}
\end{destacado}
