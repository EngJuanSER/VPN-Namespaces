\section{Implementación de Calidad de Servicio (QoS)}

\subsection{Políticas de QoS Implementadas}

Se implementaron exitosamente tres técnicas principales de QoS en ambos gateways:

\subsubsection{1. Traffic Shaping con HTB}

\begin{lstlisting}[language=bash, caption=Configuración Hierarchical Token Bucket]
# Configuración de Traffic Shaping
tc qdisc add dev wg0 root handle 1: htb default 12
tc class add dev wg0 parent 1: classid 1:1 htb rate 10mbit

# Clases de servicio diferenciadas
tc class add dev wg0 parent 1:1 classid 1:10 htb rate 3mbit ceil 5mbit   # Alta prioridad
tc class add dev wg0 parent 1:1 classid 1:11 htb rate 3mbit ceil 7mbit   # Media prioridad  
tc class add dev wg0 parent 1:1 classid 1:12 htb rate 4mbit ceil 10mbit  # Baja prioridad
\end{lstlisting}

\subsubsection{2. Traffic Classification con DSCP}

\begin{lstlisting}[language=bash, caption=Marcado DSCP para priorización]
# Marcado DSCP para diferentes tipos de tráfico
iptables -t mangle -A OUTPUT -p tcp --dport 22 -j DSCP --set-dscp 46    # SSH - Alta prioridad
iptables -t mangle -A OUTPUT -p icmp -j DSCP --set-dscp 46               # ICMP - Alta prioridad
iptables -t mangle -A OUTPUT -p tcp --dport 80 -j DSCP --set-dscp 26     # HTTP - Media prioridad
iptables -t mangle -A OUTPUT -p tcp --dport 443 -j DSCP --set-dscp 26    # HTTPS - Media prioridad
\end{lstlisting}

\subsubsection{3. Congestion Control Avanzado}

\begin{lstlisting}[language=bash, caption=Control de congestión y queue management]
# Configuración de SFQ (Stochastic Fair Queuing)
tc qdisc add dev wg0 parent 1:10 handle 10: sfq perturb 10
tc qdisc add dev wg0 parent 1:11 handle 11: sfq perturb 10
tc qdisc add dev wg0 parent 1:12 handle 12: sfq perturb 10

# Optimización TCP BBR
echo 'net.core.default_qdisc=fq' >> /etc/sysctl.conf
echo 'net.ipv4.tcp_congestion_control=bbr' >> /etc/sysctl.conf
\end{lstlisting}

\subsection{Resultados de QoS}

\begin{destacado}[Métricas de QoS Obtenidas]
\begin{itemize}
    \item \textbf{Ancho de banda controlado}: Límite efectivo de 10 Mbps aplicado
    \item \textbf{Priorización funcional}: Tráfico SSH/ICMP con máxima prioridad
    \item \textbf{Equidad de flujos}: SFQ garantiza distribución justa de recursos
    \item \textbf{Latencia optimizada}: Control de congestión BBR reduce bufferbloat
\end{itemize}
\end{destacado}

\begin{lstlisting}[language=bash, caption=Estadísticas de tráfico capturadas]
=== Estadísticas de Clases HTB ===
Clase 1:10 (Alta Prioridad): 12,120 bytes (202 paquetes)
Clase 1:12 (Baja Prioridad): 27,416 bytes (328 paquetes)
Total procesado: 39,536 bytes (530 paquetes)

=== Verificación DSCP ===
 Marcado DSCP activo para SSH (DSCP 46)
 Marcado DSCP activo para HTTP/HTTPS (DSCP 26)
 Clasificación automática funcionando
\end{lstlisting}
