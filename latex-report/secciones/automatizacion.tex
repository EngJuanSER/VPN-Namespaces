\section{Automatización e Infraestructura como Código}

\subsection{Scripts de Automatización}

\subsubsection{Script Maestro}

El script \texttt{setup\_network.sh} automatiza completamente el despliegue:

\begin{lstlisting}[language=bash, caption=Funciones principales del script maestro]
#!/bin/bash
# Script maestro para configurar toda la red VPN

# Función para instalar dependencias automáticamente
install_dependencies() {
    local container=$1
    docker exec $container bash -c "
        apt-get update > /dev/null 2>&1 && \
        apt-get install -y wireguard-tools iptables iproute2 iputils-ping > /dev/null 2>&1
    "
}

# Verificación de contenedores
verify_containers() {
    for container in gateway-a gateway-b cliente-a cliente-b-vod-server cliente-remoto; do
        if ! docker ps | grep -q $container; then
            error "Contenedor $container no está ejecutándose"
        fi
    done
}

# Función principal
main() {
    verify_containers
    install_dependencies_all
    generate_wireguard_keys
    configure_vpn
    setup_qos
    setup_security_ai
    validate_connectivity
}
\end{lstlisting}

\subsubsection{Script de Validación}

\begin{lstlisting}[language=bash, caption=Validación automática de configuración]
#!/bin/bash
# Script de validación de configuración VPN

# Verificar claves públicas
gateway_a_real_key=$(docker exec gateway-a wg show | grep "public key" | awk '{print $3}')
gateway_b_real_key=$(docker exec gateway-b wg show | grep "public key" | awk '{print $3}')

# Verificar conectividad
docker exec gateway-a ping -c 1 10.0.0.2 >/dev/null 2>&1
check_result $? "Conectividad Gateway-A a Gateway-B"

docker exec cliente-remoto ping -c 1 10.0.0.1 >/dev/null 2>&1
check_result $? "Conectividad Cliente-Remoto a Gateway-A"
\end{lstlisting}

\subsection{Beneficios de la Automatización}

\begin{destacado}[Ventajas Implementadas]
\begin{itemize}
    \item \textbf{Reproducibilidad}: El mismo resultado en cualquier entorno
    \item \textbf{Velocidad}: Configuración completa en menos de 2 minutos
    \item \textbf{Seguridad}: Generación automática de claves únicas
    \item \textbf{Validación}: Verificación automática de configuraciones
    \item \textbf{Troubleshooting}: Diagnóstico integrado y corrección automática
\end{itemize}
\end{destacado}
