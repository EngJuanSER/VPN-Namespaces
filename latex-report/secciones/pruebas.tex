\section{Pruebas y Validación}

\subsection{Comandos de Operación}

\begin{lstlisting}[language=bash, caption=Comandos principales del sistema]
# Despliegue completo automatizado
./setup_network.sh

# Validación de configuración
./validate_configuration.sh

# Sistema de IA de seguridad
./setup_security_ai.sh setup
./setup_security_ai.sh reports
./setup_security_ai.sh test

# Monitoreo del sistema
docker exec gateway-a wg show
docker exec gateway-a iptables -t nat -L
\end{lstlisting}

\subsection{Resultados de Pruebas}

\begin{exito}[Estado Final del Sistema]
\begin{itemize}
    \item \textbf{VPN Site-to-Site}: $\checkmark$ 100\% funcional entre oficinas
    \item \textbf{VPN Remote Access}: $\checkmark$ Acceso completo desde cliente remoto
    \item \textbf{QoS}: $\checkmark$ Tres políticas implementadas y validadas
    \item \textbf{IA Seguridad}: $\checkmark$ Sistema completo operativo
    \item \textbf{Automatización}: $\checkmark$ Despliegue en un solo comando
    \item \textbf{Monitoreo}: $\checkmark$ Validación continua automática
\end{itemize}
\end{exito}

\subsection{Matriz de Conectividad}

\begin{table}[H]
\centering
\caption{Matriz de conectividad verificada}
\label{tab:optimizaciones}
\begin{tabular}{|l|l|c|l|}
\hline
\textbf{Origen} & \textbf{Destino} & \textbf{Resultado} & \textbf{Comentarios} \\
\hline
cliente-remoto & gateway-a (10.0.0.1) & $\checkmark$ Exitoso & Conectividad VPN básica \\
\hline
cliente-remoto & cliente-a (172.16.10.2) & $\checkmark$ Exitoso & Acceso a red Oficina A \\
\hline
cliente-remoto & cliente-b (172.16.20.2) & $\checkmark$ Exitoso & Acceso a red Oficina B \\
\hline
cliente-a & cliente-b (172.16.20.2) & $\checkmark$ Exitoso & Site-to-Site funcional \\
\hline
cliente-b & cliente-a (172.16.10.2) & $\checkmark$ Exitoso & Bidireccional confirmado \\
\hline
\end{tabular}
\end{table}

\subsection{Resolución de Problemas}

Durante la implementación se enfrentaron y resolvieron varios desafíos técnicos:

\begin{alerta}[Problema 1: Endpoints Incorrectos]
\textbf{Síntoma}: Los túneles WireGuard se establecían pero no había tráfico bidireccional\\
\textbf{Causa}: IPs de endpoints intercambiadas entre gateways\\
\textbf{Solución}: Corrección de endpoints a las IPs reales asignadas por Docker
\end{alerta}

\begin{alerta}[Problema 2: Claves Públicas Inconsistentes]
\textbf{Síntoma}: Cliente remoto enviaba tráfico pero no recibía respuestas\\
\textbf{Causa}: Clave pública del cliente-remoto no coincidía en gateway-a\\
\textbf{Solución}: Implementación de generación automática de claves
\end{alerta}

\begin{alerta}[Problema 3: Inconsistencias de MTU]
\textbf{Síntoma}: Fragmentación de paquetes y degradación de rendimiento\\
\textbf{Causa}: MTU por defecto de 65456 en lugar del óptimo 1420\\
\textbf{Solución}: Configuración explícita de MTU = 1420 en todas las interfaces
\end{alerta}

\subsection{Optimizaciones Implementadas}

\begin{table}[H]
\centering
\caption{Optimizaciones de red implementadas}
\label{tab:optimizaciones}
\begin{tabular}{|l|p{8cm}|}
\hline
\textbf{Optimización} & \textbf{Beneficio} \\
\hline
MTU 1420 & Eliminación de fragmentación de paquetes \\
\hline
NAT específico & Mejor rendimiento vs. reglas genéricas \\
\hline
Persistent Keepalive & Mantiene conexiones a través de firewalls \\
\hline
BBR Congestion Control & Reduce latencia y mejora throughput \\
\hline
SFQ Queue Management & Equidad entre flujos de diferentes usuarios \\
\hline
\end{tabular}
\end{table}
