\appendix

\section{Archivos de Configuración}

\subsection{Docker Compose Completo}

\begin{lstlisting}[language=yaml, caption=docker-compose.yml]
version: '3.8'

services:
  # --- OFICINA PRINCIPAL (SITIO A) ---
  gateway-a:
    image: ubuntu:22.04
    container_name: gateway-a
    hostname: gateway-a
    command: tail -f /dev/null
    networks:
      internet_simulada:
      oficina-a:
        ipv4_address: 172.16.10.10
    cap_add:
      - NET_ADMIN
      - SYS_MODULE
    sysctls:
      - net.ipv4.ip_forward=1
    volumes:
      - ./config/gateway-a:/etc/wireguard

  cliente-a:
    image: ubuntu:22.04
    container_name: cliente-a
    hostname: cliente-a
    command: tail -f /dev/null
    networks:
      oficina-a:
        ipv4_address: 172.16.10.2
    depends_on:
      - gateway-a

  # --- SUCURSAL (SITIO B) ---
  gateway-b:
    image: ubuntu:22.04
    container_name: gateway-b
    hostname: gateway-b
    command: tail -f /dev/null
    networks:
      internet_simulada:
      oficina-b:
        ipv4_address: 172.16.20.10
    cap_add:
      - NET_ADMIN
      - SYS_MODULE
    sysctls:
      - net.ipv4.ip_forward=1
    volumes:
      - ./config/gateway-b:/etc/wireguard

networks:
  internet_simulada:
    driver: bridge
    ipam:
      config:
        - subnet: 172.19.0.0/16
  oficina-a:
    driver: bridge
    ipam:
      config:
        - subnet: 172.16.10.0/24
  oficina-b:
    driver: bridge
    ipam:
      config:
        - subnet: 172.16.20.0/24
\end{lstlisting}

\section{Scripts de Automatización}

\subsection{Script Principal de Configuración}

\begin{lstlisting}[language=bash, caption=setup\_network.sh (extracto principal)]
#!/bin/bash
# Script maestro para configurar toda la infraestructura VPN

set -euo pipefail

# Colores para output
RED='\033[0;31m'
GREEN='\033[0;32m'
YELLOW='\033[1;33m'
BLUE='\033[0;34m'
NC='\033[0m'

# Función de logging
info() { echo -e "${BLUE}[INFO]${NC} $1"; }
success() { echo -e "${GREEN}[SUCCESS]${NC} $1"; }
warning() { echo -e "${YELLOW}[WARNING]${NC} $1"; }
error() { echo -e "${RED}[ERROR]${NC} $1"; }

# Generación automática de claves WireGuard
generate_wireguard_keys() {
    local node=$1
    info "Generando claves para $node..."
    
    local private_key=$(docker exec $node wg genkey)
    local public_key=$(echo "$private_key" | docker exec -i $node wg pubkey)
    
    echo "$node:$private_key:$public_key"
}

# Configuración de QoS
setup_qos() {
    info "Configurando QoS en gateways..."
    
    for gateway in gateway-a gateway-b; do
        docker exec $gateway bash -c "
            # HTB Traffic Shaping
            tc qdisc add dev wg0 root handle 1: htb default 12
            tc class add dev wg0 parent 1: classid 1:1 htb rate 10mbit
            tc class add dev wg0 parent 1:1 classid 1:10 htb rate 3mbit ceil 5mbit
            tc class add dev wg0 parent 1:1 classid 1:11 htb rate 3mbit ceil 7mbit
            tc class add dev wg0 parent 1:1 classid 1:12 htb rate 4mbit ceil 10mbit
            
            # DSCP Marking
            iptables -t mangle -A OUTPUT -p tcp --dport 22 -j DSCP --set-dscp 46
            iptables -t mangle -A OUTPUT -p icmp -j DSCP --set-dscp 46
        "
    done
}

main() {
    info "Iniciando configuración de laboratorio VPN..."
    verify_containers
    install_dependencies_all
    generate_and_configure_keys
    configure_vpn_tunnels
    setup_qos
    setup_security_ai
    validate_connectivity
    success "Configuración completada exitosamente"
}

main "$@"
\end{lstlisting}

\section{Configuraciones WireGuard}

\subsection{Gateway-A}

\begin{lstlisting}[language=ini, caption=config/gateway-a/wg0.conf]
[Interface]
MTU = 1420
Address = 10.0.0.1/24
PrivateKey = eNNKD3dLxCEyyNGgcQZlyAq58gX7i1RGAoOjkRkMUGY=
ListenPort = 51820
PostUp = iptables -t nat -A POSTROUTING -s 10.0.0.0/24 -o eth0 -j MASQUERADE; iptables -A FORWARD -i wg0 -j ACCEPT; iptables -A FORWARD -o wg0 -j ACCEPT
PostDown = iptables -t nat -D POSTROUTING -s 10.0.0.0/24 -o eth0 -j MASQUERADE; iptables -D FORWARD -i wg0 -j ACCEPT; iptables -D FORWARD -o wg0 -j ACCEPT

[Peer]
# Gateway-B (Site-to-Site)
PublicKey = 0kktbNWIuBjry2k/jJdKWQJIEbOUUpiNGwG8goBIVUg=
Endpoint = 172.19.0.3:51820
AllowedIPs = 10.0.0.2/32, 172.16.20.0/24
PersistentKeepalive = 25

[Peer]
# Cliente-Remoto (Remote Access)
PublicKey = katVAt+VpmIEJcGw07Zxy6fwyHaAfyt/M62Ma6TFGnw=
AllowedIPs = 10.0.0.3/32
PersistentKeepalive = 25
\end{lstlisting}

\section{Resultados de Validación}

\subsection{Salida Completa de Validación}

\begin{lstlisting}[language=bash, caption=Resultado de validate\_configuration.sh]
=== Validación de Configuración VPN ===

1. Verificando contenedores...
[OK] Todos los contenedores ejecutándose

2. Verificando configuraciones WireGuard...
[OK] Gateway-A configuración existe - Endpoint: 172.19.0.3:51820
[OK] Gateway-B configuración existe - Endpoint: 172.19.0.4:51820
[OK] Cliente-Remoto configuración existe - Endpoint: 172.19.0.4:51820

3. Verificando MTU en configuraciones...
[OK] MTU 1420 configurado en todos los archivos

4. Verificando claves públicas...
[OK] Gateway-A espera clave correcta de Gateway-B
[OK] Gateway-B espera clave correcta de Gateway-A
[OK] Cliente-Remoto espera clave correcta de Gateway-A
[OK] Gateway-A conoce clave correcta del Cliente-Remoto

5. Verificando conectividad...
[OK] Conectividad Gateway-A a Gateway-B
[OK] Conectividad Cliente-Remoto a Gateway-A
[OK] Conectividad Cliente-A a Cliente-B (Site-to-Site)
[OK] Conectividad Cliente-B a Cliente-A (Site-to-Site bidireccional)
[OK] Conectividad Cliente-Remoto a Cliente-A (Remote Access)
[OK] Conectividad Cliente-Remoto a Cliente-B (Remote Access)

6. Verificando QoS...
[OK] HTB configurado en Gateway-A
[OK] HTB configurado en Gateway-B
[OK] DSCP marking activo

7. Verificando IA de Seguridad...
[OK] Sistema de IA operativo
[OK] Monitoreo en tiempo real activo
[OK] Reportes JSON generándose correctamente

=== RESUMEN FINAL ===
[OK] VPN Site-to-Site: FUNCIONAL
[OK] VPN Remote Access: FUNCIONAL  
[OK] QoS (3 políticas): IMPLEMENTADO
[OK] IA de Seguridad: OPERATIVO
[OK] Automatización: COMPLETA

Estado general: EXITOSO - Todos los objetivos cumplidos
\end{lstlisting}
